% \documentclass{emulateapj}
\documentclass[letterpaper,12pt,preprint]{aastex}

% has to be before amssymb it seems
%\usepackage{color,hyperref}
%\definecolor{linkcolor}{rgb}{0,0,0.5}
%\hypersetup{colorlinks=true,linkcolor=linkcolor,citecolor=linkcolor,
%            filecolor=linkcolor,urlcolor=linkcolor}
%\usepackage{amssymb,amsmath}

\usepackage{color}
\usepackage{url}
\usepackage{graphicx}
\graphicspath{{figures/}}

% For Python code
\usepackage{listings}
\definecolor{lbcolor}{rgb}{0.9,0.9,0.9}
\lstset{language=Python,
        basicstyle=\footnotesize\ttfamily,
        showspaces=false,
        showstringspaces=false,
        tabsize=2,
        breaklines=false,
        breakatwhitespace=true,
        identifierstyle=\ttfamily,
        keywordstyle=\bfseries\color[rgb]{0.133,0.545,0.133},
        commentstyle=\color[rgb]{0.133,0.545,0.133},
        stringstyle=\color[rgb]{0.627,0.126,0.941},
    }

% Draft watermark:
%\usepackage{draftwatermark}
%\SetWatermarkLightness{0.9}
%\SetWatermarkScale{4}

% Some macros
\newcommand{\todo}[1]{{\color{red} [TODO: #1]}}
\newcommand{\foreign}[1]{{\it #1}}

\newcommand{\apriori}{\foreign{a priori}}
\newcommand{\adhoc}{\foreign{ad hoc}}
\newcommand{\etal}{\foreign{et\,al.}}
\newcommand{\etc}{\foreign{etc.}}

\newcommand{\Fig}[1]{Figure~\ref{fig:#1}}
\newcommand{\fig}[1]{\Fig{#1}}
\newcommand{\figlabel}[1]{\label{fig:#1}}
\newcommand{\Eq}[1]{Equation~(\ref{eq:#1})}
\newcommand{\eq}[1]{\Eq{#1}}
\newcommand{\eqs}[2]{Equations~(\ref{eq:#1})-(\ref{eq:#2})}
\newcommand{\eqlabel}[1]{\label{eq:#1}}
\newcommand{\Sect}[1]{Section~\ref{sect:#1}}
\newcommand{\sect}[1]{\Sect{#1}}
\newcommand{\sects}[1]{Sections~#1}
\newcommand{\App}[1]{Appendix~\ref{sect:#1}}
\newcommand{\app}[1]{\App{#1}}
\newcommand{\sectlabel}[1]{\label{sect:#1}}

\usepackage[normalem]{ulem}
\newcommand{\new}[1]{{\color{red} #1}}
\newcommand{\old}[1]{{\sout{#1}}}


\begin{document}

\title{Stripe 82 Deep Dive: RR Lyrae Past the Survey Limit}

\newcommand{\escience}{1}
\newcommand{\columbia}{2}
\newcommand{\nyu}{3}
\newcommand{\cds}{4}
\newcommand{\mpia}{5}
\newcommand{\uwastro}{6}

\author{Jacob T. VanderPlas\altaffilmark{\escience}, 
	    Adrian M. Price-Whelan\altaffilmark{\columbia},
	    David W. Hogg\altaffilmark{\nyu,\cds,\mpia},
	    {\v Z}eljko Ivezi{\'c}\altaffilmark{\uwastro}}

\altaffiltext{\escience}{eScience Institute, University of Washington}
\altaffiltext{\columbia}{Department of Astronomy, Columbia University, 550 W 120th St., New York, NY 10027, USA}
\altaffiltext{\nyu}{Center for Cosmology and Particle Physics, Department of Physics, New York University, 4 Washington Place, New York, NY, 10003, USA}
\altaffiltext{\cds}{Center for Data Science, New York University, 4 Washington Place, New York, NY, 10003, USA}
\altaffiltext{\mpia}{Max-Planck-Institut f\"ur Astronomie, K\"onigstuhl 17, D-69117 Heidelberg, Germany}
\altaffiltext{\uwastro}{Department of Astronomy, University of Washington}

\begin{abstract}
% Context
At distances larger than $\approx$30 kpc, RR Lyrae stars show strong spatial inhomogeneity and have been shown to trace accreted structures in the Galactic halo. Simulations predict that stars in the extreme halo --- more than 100 kpc in Galactocentric radius --- are predominantly associated with unrelaxed, accreted structures such as tidal streams, shells, and the progenitor stellar systems themselves. Such features have been observed at large distances around a select few nearby galaxies, however it is yet unknown what structures our own Galaxy hosts in the outer halo.
% Aims
In this work, we aim to discover RR Lyrae stars at distances larger than 100 kpc with the hope and expectation that these stars are associated with interesting accreted structures in the halo of the Milky Way. 
% Methods
We use reprocessed photometry of the Stripe 82 region of the Sloan Digital Sky Survey and push the detection limit of RR Lyrae stars further both through this more precise photometry and less restrictive modeling of the light curves. In particular, since RR Lyrae stars exhibit $\approx$1 mag variability in the SDSS bandpasses, by using a likelihood-based, multi-band model for period determination we do not require detection of the source at all epochs.
% Results
We find XX RR Lyrae stars ...
% Conclusions
Huh?

\end{abstract}

\section{Introduction}
\sectlabel{introduction}

Some relevant references are \citet{Vanderplas2015}, \citet{Sesar2010}, \citet{Oluseyi12}.

\section{Methods}
\sectlabel{methods}

\section{Results}
\sectlabel{results}

\section{Discussion}
\sectlabel{discussion}

\section{Conclusions}
\sectlabel{conclusion}

\bibliographystyle{apj}
\bibliography{paper}

\end{document}
